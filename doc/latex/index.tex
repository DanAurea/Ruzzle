Le Ruzzle est un jeu originaire du monde mobile, ce jeu consiste à former le plus de mots en un temps imparti à partir d'une grille de lettres générée aléatoirement. Le but étant de marquer le plus de points possible pour celà des bonus lettre et mot double/triple sont répartis sur la grille mais il n'est pas permis de réutiliser deux fois la même case.

\subsection*{Optimisation}

Pour le moment le solver vérifiant toutes les combinaisons en ouvrant le dictionnaire correspondant et en vérifiant toutes les possibilités, les temps d'accés rendent l'exécution un peu lente ($\sim$ 25s max ). Il reste néanmoins quelques modifications à apporter pour réduire au maximum ce temps.

\subsection*{Instructions de compilation}

\begin{quote}
\$ make \end{quote}


Permet de compiler l'ensemble des sources, l'exécutable généré peut être retrouvé dans $\ast$$\ast$./bin$\ast$$\ast$ .

\begin{quote}
\$ make mrproper \end{quote}


Permet de nettoyer le dossier.

\subsection*{Utilisation}

\begin{quote}
\$ ./bin/ruzzle\+Solver \end{quote}


Permet d'exécuter le programme en générant une grille aléatoire, les lettres sont tirées au hasard en prenant compte de leurs fréquences d'apparition dans la langue française.

On peut prédéfinir une grille à l'aide d'une chaine de 16 caractères.

Par exemple\+:

\begin{quote}
\$ ./bin/ruzzle\+Solver adcksxirmdesuckh \end{quote}


Génèrera la grille \+:

\begin{quote}
A~~~~D~~~~C~~~~K

S~~~~X~~~~I~~~~R

M~~~~D~~~~E~~~~S

U~~~~C~~~~K~~~~H

\end{quote}


Dans le cas d'une chaîne trop petite un message d'erreur apparaîtra, si la chaîne est trop grande, la grille sera composée des 16 premiers caractères.

Les bonus lettres et mots sont tirés aléatoirement de manière à ce qu'il n'y ait pas trop de bonus.

Le résultat du solver se retrouve dans \+:

\begin{quote}
assets/list\+Words.\+txt \end{quote}


\subsection*{Documentation}

Il est possible de se documenter sur la structure du solver grâce à une documentation générée par doxygen. Le fichier de configuration pour la documentation est fournie et permet donc de régénérer la documentation en cas de soucis, une page html sera crée reprenant toute la documentation \+:

\begin{quote}
doc/html/index.\+html\end{quote}
